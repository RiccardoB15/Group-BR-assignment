\documentclass[a4paper, 12pt, english]{article}
\usepackage[onehalfspacing]{setspace}
\usepackage[utf8]{inputenc}
\usepackage[T1]{fontenc}
\usepackage{graphicx, caption}
\usepackage{natbib}
%\bibliographystyle{apa}
\bibliographystyle{apalike2}
%\setcitestyle{authoryear,open={((},close={))}}
\usepackage{multicol,lipsum}
{\par\nobreak\vfil\penalty0\vfilneg
	\vtop\bgroup}
{\par\xdef\tpd{\the\prevdepth}\egroup
	\prevdepth=\tpd}
\usepackage{adjustbox}
%\usepackage[backend=biber,style=science,citestyle=authoryear]{biblatex}
%\addbibresource{library.bib}
\usepackage[hidelinks]{hyperref}
\usepackage{booktabs}
\usepackage{dcolumn}
\usepackage{csquotes}
\usepackage{svg}
\usepackage{caption}
\usepackage{subcaption}
\usepackage{wrapfig}
\usepackage{float}
\usepackage{geometry}
\usepackage{blindtext}

%\usepackage[backend=biber,style=alphabetic,sorting=ynt]{biblatex}

%\addbibresource{library.bib}


\AtBeginDocument{\AtBeginShipoutNext{\AtBeginShipoutDiscard}}
\geometry{left=1in,right=1in,top=1.5in,bottom=1.5in}
\begin{document}
		
\title{Group Assignment DtfF}		
\author{Beatrice Fontana and Riccardo Barbiero}

\date{\today}
\maketitle
Matricula Number:
    \begin{itemize}
        \item 19-753-854 - Beatrice Fontana
        \item 19-753-862 - Riccardo Barbiero 
    \end{itemize}  
    
\begin{abstract}
\blindtext[1]
\\
\\
\\
\end{abstract}

\newpage
\pagenumbering{Roman}


% table of contents
%\setcounter{tocdepth}{2}
\tableofcontents
%\vspace{2cm}
%\newpage
%\listoffigures
%\clearpage
%\listoftables

\clearpage

\pagenumbering{arabic}
\section{Introduction}
\blindtext[1]



\clearpage
\section{Literature}
\blindtext[1] (\cite{clift2012economic}, \cite{morgan2012supporting}) and economical decisions.
For example, consumers are influenced in their decision making by patriotism \citep{spielmann2018product}. 
They tend to buy domestic products despite the availability of imported otherwise identical products \citep{shankarmahesh2006consumer}.\\


\subsection{More about literature}

It may be argued that our dependent variable is not a perfect measure. However, it represents a good proxy for the individual preference level between national and international equities. Still, following \cite{morse2011patriotism}, we use an alternative measure for home bias by incorporating the optimal portfolio suggested by the CAPM using the world market capitalization. The CAPM home bias \% is computed as the difference between the optimal CAPM domestic weight for each country and the holdings of domestic equities:
\begin{equation}
     CAPM \: home \: bias\: \% = \frac{home\: investment}{home + foreign\: investment} - \frac{home\: capitalization}{world\: capitalization}
\end{equation}
Results using the alternative home bias measure are consistent with the ones provided by HOME-BIAS and are reported in the robustness check section.\\

%Means of Home bias 2 per Country
\begin{table}[!htpb] 
    \centering
    \caption{Mean of CAPM home bias for each country.}
    \label{Tab:MeanHomeBias}
    
     \begin{tabular}{lr} 
        \toprule
        Country & Mean\\
        \midrule
             Estonia & 0.25 \\
             Germany & 0.65 \\
             China & 0.44  \\
             Hong Kong & 0.41 \\
             Japan & 0.45 \\
             Taiwan & 0.45 \\
             Vietnam & 0.56\\
        \bottomrule
    \end{tabular}\\  
\end{table}
\newpage

\section{Method}
\blindtext[1]

\newpage
\section{Results}
%to insert tables and figures
\blindtext[1]

\clearpage
\section{Conclusion}
\blindtext

\clearpage
\newpage
\bibliography{test}
%\bibliography{test}
%\printbibliography
\newpage
\appendix
% \\
\end{document}
